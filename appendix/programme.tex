\chapter{Programme}\label{apdx: programme}
In this chapter, we briefly talk about the programme we create for this report. This chapter acts as a high level \verb+README+ file for our programme. The whole project can be found using the link to our GitHub repository \url{https://github.com/LinusLin6769/DCLaboratory01}.

\section{How the programme works}
The usage of the programme goes as the following:
\begin{enumerate}
    \item Set how the experiment is intended to be done using the \verb+config.json+ file.
    \item Run the experiment, which will generate experimental data.
    \item Investigate the generated data.
\end{enumerate}

\section{Experiment configurations}
The things that are controlled by the \verb+config.json+ file include the following:
\begin{enumerate}
    \item Whether this run is for real, or it is just a test of something during development.
    \item Which dataset to use and which time series to run on.
    \item Transformation configuration: how the transformation will be performed and its hyperparameter space.
    \item Modelling configuration: how the modelling process is done, e.g., validation and test size, retrain window size.
    \item What are the models to be included in the experiment.
    \item How many CPU cores to use for the implemented multi-processing.
\end{enumerate}

\section{Running the experiment}
The experiment runs by calling
\begin{verbatim}
    python3 main.py
\end{verbatim}
from the command-line tool. \verb+main.py+ calls the \verb+config.json+ file, where the user controls how the experiment should go at the beginning, and the output is a directory created under the \verb+experiment_info/+ or \verb+test_experiment_info/+ directory depending on whether this is a real run or a test during development. The created directory is named after the starting time of the experiment. For every model trained, all the series are used. A \verb+.JSON+ file containing the experiment information of this model is created and placed in the experiment's directory before the programme moves on to another model until all models are trained.

\section{Analysing the results}
To analyse the experiment, one can use a Jupyter Notebook or any other tools to read the data generated by the experiment.

\section{DC Transformation programme}
As the core of this experiment, the programme of Directional Change Transformation is implemented as an independent class called \verb+DCTransformer+ in \verb+dc_transformation.py+ file. It is easy to use for those interested, and demonstrations can be found at the bottom of \verb+dc_transformation.py+.
