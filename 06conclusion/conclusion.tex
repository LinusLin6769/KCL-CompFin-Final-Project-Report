\chapter{Conclusion}\label{ch: conclusion}
This chapter concludes our report with two paragraphs: we first outline the contributions we have made in this report and then discuss possible extensions to our work.

On the higher level, this is an exploratory report - we present an original idea of bringing the Directional Change Framework and target transformation techniques together for univariate time series forecasting. The report also covers a robust experiment and critical analyses of the proposed methodology. In particular, there are several highlights to this report. First, we comprehensively discuss how target transformation techniques work in univariate time series forecasting. We cover both the intuitions and technical details on the level that can be generalised to most Machine Learning univariate time series regression modelling cases. Secondly, in addition to addressing the DC framework, we present the pseudocode Algorithm \ref{alg: dc} for the DC event identification algorithm. Although there are other versions of pseudocode for such an algorithm in other publications, they normally are not for general purposes, i.e., their versions are for specific tasks. For example, the one in Glattfelder et al. (\citeyear{glattfelder2011patterns}) is for counting the number of directional change events and the one in Golub et al. (\citeyear{golub2016multi}) is for measuring the overshoots with a $\delta$ threshold. On the other hand, Algorithm \ref{alg: dc} is for general purposes - it marks the states of the input time series and can then be further extended for any other usages. Thirdly, we devise a novel target transformation method based on the DC framework and interpolation methods. Our innovation demonstrates a comprehensive understanding of both methodologies. Finally, we present interesting insights on DC Transformation according to our large-scale experiment, critical analyses, and robust scientific practice.

For prospective future studies on DC Transformation, we reckon several directions are worth further investigation. First, better estimates of the threshold value $\delta$ can be devised. Additionally, more complex methods associated with such parameters can be explored. For example, instead of deciding on a $\delta$ to use, Alpha Engine utilises four different sets of threshold values at the same time to gain more insights into the intrinsic events (see Golub et al. (\citeyear{golub2018alpha})). Secondly, as we tune the $\delta$ value as a hyperparameter, information on why the model chose a $\delta$ value for each time series can be interesting. This can be a potential source of information to further characterise a time series. Thirdly, more sophisticated methods for reconstructing the time series can be devised. DCT simply interpolates the values extracted by the DC algorithm, but perhaps more intelligent mechanisms can better take advantage of the information extracted by the DC algorithm. Fourth, in light of the DC framework research mostly focusing on high-frequency data, datasets with higher frequencies should be tested, e.g., ten-minute or tick-by-tick datasets. We think it is very likely that high-frequency data is where DCT can fully live up to its potential. Finally, other combinations of forecasting objectives and models should be experimented to expand the horizon of our understanding of DC Transformation.
