\chapter{Introduction}
The primary objective of this report is to propose and evaluate the Directional Change Transformation (DCT), which is a noval target transformation method for time series forecasting based on the Directional Change (DC) Intrinsic Time Framework. Target transformation is a preprocessing method that is often used as an additional layer within time series modelling procedures. Instead of operating on the feature space of a regression problem like typical feature engineering methods do, target transformation takes a step further and operates on the target. Metaphorically, preprocessing the feature space in modelling is like organising the ingredients to make the cooking easier for the chef. On the other hand, operating on the target is like having a better recipe such that is easier for the chef to cook something delicious. Due to such nature, target transformation is also referred to as target statement transformation. DC Intrinsic Time Framework is a tool used to study financial time series, especially high frequency time series. It provides noval insights that have been utilised by trading strategies and many market analyese. Learning its ability to extract information from time series motivates us to turn it into a target transformation method - we hope it can provide the models with a better target statement to learn from in time series forecasting tasks.

In the next chapter we give comprehensive background theory on the topics related to our proposition. We address time series forecasting with machine learning regression models and many general notions regarding the procedure. Building on the introduction, we then cover how target transformation methods can be embedded into the general procedure of machine learning modelling, and we also discuss important considerations of implementing target transformation methods. Last, but by no means least, we elaborate the DC Intrinsic Time Framework in-depth and also provide a detailed pseudocode for the directional change event identification algorithm. In Chapter \ref{ch: literature review}, we survey the important literature regarding target transformation and the DC framework that established the ground for our exploration of bringing the two together. In Chapter \ref{ch: methodology}, we elaborate the technical details of DC Transformation and the experiement we used to investigate DCT. In Chapter \ref{ch: results and eval}, we present the results yielded from our experiment and the analyses about DCT. Finally, a conclusion is provided in Chapter \ref{ch: conclusion}.
