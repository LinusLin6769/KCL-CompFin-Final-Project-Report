\chapter{Introduction}
The primary objective of this report is to propose and evaluate the Directional Change Transformation (DCT), a novel target transformation method for time series forecasting based on the Directional Change (DC) Intrinsic Time Framework. Target transformation is a preprocessing method often used as an additional layer within time series modelling procedures. Instead of operating on the feature space of a regression problem like typical feature engineering methods, target transformation takes a step further and operates on the target to improve modelling performance. Metaphorically, preprocessing the feature space in modelling is like organising the ingredients to make cooking a dish easier for the chef. On the other hand, operating on the target is like having a better recipe. Due to such nature, target transformation is also referred to as target statement transformation. DC Intrinsic Time Framework is used to study financial time series, especially high-frequency time series. It provides novel insights utilised by trading strategies and many market analyses. Learning its ability to extract information from time series motivates us to turn it into a target transformation method - we hope it can provide models with a better target statement to learn from in time series forecasting tasks.

In the next chapter, we give a comprehensive background theory on three topics related to our proposition. First, we address time series forecasting with machine learning regression models and many general notions regarding the procedure. Secondly, building on the introduction, we cover how target transformation methods can be embedded into the general procedure of machine learning modelling; we also discuss important considerations for implementing target transformation methods. Thirdly, we elaborate on the DC Intrinsic Time Framework in-depth and provide detailed pseudocode for the directional change event identification algorithm. In Chapter \ref{ch: literature review}, we survey the important literature regarding target transformation and the DC framework that established the ground for our exploration of bringing the two together. In Chapter \ref{ch: methodology}, we elaborate on the technical details of the proposed DC Transformation and the experiment we used to investigate its characteristics. In Chapter \ref{ch: results and eval}, we present the results yielded from our experiment and the analyses about DCT. Finally, a conclusion is provided in Chapter \ref{ch: conclusion}.
